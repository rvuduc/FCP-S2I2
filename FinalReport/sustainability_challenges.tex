\subsection{Assessment of Software Sustainability Challenges}
Some sustainability challenges are common across all of software development, some are specific to
disruptive future platforms, and yet others are unique to our application communities. We feel that
one key to sustainability is to cultivate a wide, engaged community. Members of that community
should be able to work on different aspects of grand challenge problems without feeling overly
burdened by other members’ needs or schedules. This conceptualization effort will find methods to
ease sharing while catalyzing loosely coupled scientific investigation.

\paragraph{Common Challenges:} To start a community, we will find common organizational needs and
issues across our application areas. We will identify which phases of development benefit from
or require a center’s resources. Examples may include common software and data distribution
repositories, mechanisms for tracking and attributing data and software (re)use through provenance,
providing expert guidance on tuning and debugging issues, or providing writing assistance for
documentation.
We must also overcome resistance to sharing efforts. Scientists want to spend their time on
science, not on organization, and reasonably resist top-down efforts to force sharing and coordination.

We will survey and investigate methods for low-overhead community-wide sharing, such as agile
development techniques [42, 60], lightweight and serendipitous discussion tools like wikis [174, 93],
software development social networking [97, 75], or even venues for publishing communitywide
blurbs on daily activity [128]. Having visibility into others’ development encourages social
engagement and communities [167]. The center activities will also assist the partner SSE’s and
SSI’s that form focused pools of domain researchers.

\paragraph{Future Platform Challenges:} Platforms and accelerators successful in the market are positively
disruptive and open new scientific vistas. Platforms that disappear from the market are negatively
disruptive and may waste scientists’ development efforts. A scientist also risks lacking a peer group
when pioneering use of a future platform. In addition, portability of applications across competing
platforms may be limited due to the proprietary nature of emerging accelerators.
We will determine how best to reduce adoption risk both in our application community and
beyond. We will investigate protocols to engage vendors and find mutually acceptable archival
procedures to ensure documentation and support materials are available without requiring explicit
vendor support past a product’s commercial lifetime. A center can also house software for uniformly
composing applications that operate across hybrid accelerator platforms, and incubate community
standards. Other challenges include quickly sharing platform knowledge within the community
and also developing procedures for evaluating and recommending future platforms to researchers.

Solving these challenges may include collaborating with vendors to provide access to demonstration
and debugging systems, assisting with evaluating platforms for specific science needs, and helping
evaluate a software’s potential for becoming sustainable.

\paragraph{Application Area Challenges:} Each of our application areas is strongly data-driven. Securely
providing and sharing scientifically useful data to analyze across our communities is a significant
challenge. We will evaluate community needs and investigate software infrastructure necessary
to reduce the burden of discovering and sharing curated data and metadata. Possible support a
center can provide include quality checking, such as fitting anonymous, artificial data models to sensitive real-world data, coordinating the use of private data across research groups, and tracking
provenance of data creation and reuse for attribution and verification.
Data-driven software often uses task-specific data formats and internal data structures that may
not satisfy a larger community. We will investigate appropriate balances for task-tuned data access
methods against common access methods. We will engage application area communities like the
Data-Enabled Life Science Alliance (DELSA) both to facilitate data sharing and to bring relevant
SSI and SSE artifacts to a larger audience.

Finally, we must address the challenge of matching science needs and applications to future
platforms and accelerators. We will investigate methods reasonable both to commercial vendors
and scientists for evaluating combinations of platforms and applications with a goal of producing
better scientific results more easily. The investigators have a long history of successful evaluations
for vendors and applications [8, 30, 20, 21, 22, 138, 9, 175, 90, 26, 34, 108, 36, 24, 28, 25, 112].
