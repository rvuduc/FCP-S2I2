\subsection{Conceptualization Strategic Plan}
Our approach for assembling the conceptual design is iterative. Each workshop will expose the
grand challenges and software infrastructure design issues, and incrementally inform the design
document. Cross-cutting issues will be discussed after the domain workshop series. Here we explain
the design document’s coverage and our approach to assembling the information.

The proposed conceptualization activities will focus around the following thrusts: (1) understanding
the software requirements due to future computing platforms and of application that will
use these platforms; (2) exploring the related software development, deployment and sustainability
challenges and requirements; (3) the necessary modalities and mechanisms for training, outreach
and community engagement; and (4) exploring structural and management models of our center.

\paragraph{Discovering Application Requirements} This effort’s goal is to understand the scientists’ software
infrastructure requirements to empower scientific communities to leverage the disruptive
potential of future computing platforms toward grand challenge problems, and to explore models
and mechanism for sustaining this infrastructure. As part of the conceptualization efforts, we will
engage with key researchers in the biological, social, and security science domains. Our proposed
steering committee includes leaders in these fields, and we will leverage their expertise to identity
relevant community members. We will then host focused workshops to understand how these
communities can most effectively exploit emerging and future computing platforms, and existing
software-related challenges and barriers. Our goal will be to identify critical cross-cutting software
elements and frameworks where sustained effort can drive innovations in these fields and improve
the productivity of the researchers. We will also use “Birds of a Feather” sessions in key conferences
to reach out to the broader community in these application areas.

\paragraph{Characterizing Software Requirements and Practices} The second thrust of our conceptualization
activity focuses on ascertaining software requirements in terms of functionality as well as the
development, deployment and sustainability challenges. This process requires exploring current
practices in the targeted application communities. This includes understanding what software
systems (e.g. tools, models, algorithms, libraries, frameworks and middleware) are in use by these
communities, how these systems are developed and maintained, and existing best practices as well
as gaps. We will identify the key members of the communities through our steering committee
and work with them to list critical software, and current best practises and processes. Following
this, we will develop common models and mechanisms to support, sustain, and extend critical
and cross-cutting software infrastructure. We will involve the steering committee and key domain
members throughout this process and develop mechanisms both consistent with current practice and
responsive to new requirements.

\paragraph{Understanding Training/Outreach Mechanisms} The proposed center must also manage community
outreach and training for software and future platforms. Identifying the appropriate modalities
and mechanisms for engaging with communities on a continuous basis is critical. During
conceptualization activities, we will work with our steering committee and our community members
to isolate best ways to serve the science communities’ training needs. We will develop strategies
through community discussion and polls to guide tutorial topics. These tutorials will be offered at
community events, and training material made available for outreach to under-represented groups.

\paragraph{Exploring Models for Envisioned S2I2} Our conceptualization’s final step focuses on exploring
models for an S2I2 center that can server as a long-term hub of excellence. Specifically, we will
examine organizational and governance structures, required infrastructure for software dissemination
and management, mechanisms for community engagement involving both software developers
as well as users (including partners from SSIs and SSEs), and sustainability models for both the
software infrastructure and the center. For this activity, we will study existing centers like NSFfunded
ERCs, STC, and I/UCRCs. We will also explore models used in the open source community
(e.g. Apache) and industry (e.g. RedHat) for community engagement and software sustainability.
The steering committee and key community members will be involved throughout this process to
maintain consistency with current community practices.

The conceptualization effort will produce a conceptual design addressing the following:
\begin{itemize}
\item Relative priorities of research objectives and grand challenges: What are the center’s best
challenge problems for sustained impact? Which activities are sequential and concurrent?
\item Functional requirements and macro design of software system: What are the common
functional requirements? What existing software components can be adopted after hardening?
Where are the gaps? How is the functionality expanded to meet research objectives?
\item Organization and management structure: Is it a physical or a virtual center? How do the PI
institutions interface? How do the technology and domain researchers interface? What are
the staff roles necessary (e.g. development, documentation, outreach)?
\item Interactions with other SI2 groups: How does the center support and interact with SSE/SSIs?
\item Research infrastructure: What is the necessary CI? What existing resources can be leveraged
at the center, its member locations, SSE/SSIs, and other NSF facilities?
\item Budget and resource allocation: What should the funding levels be? What are the budgetary
headlines, and allocation to the staff roles and members of the center? What is the strategy
for growth and sustenance?
\item Schedule for implementation: What is the timeline for setup, growth? What is the exit criteria?
\item Risk analysis and mitigation: How do we measure success? What are the alternative paths?
What is the broader impact?
\end{itemize}
