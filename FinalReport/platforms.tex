\subsection{Potential of Future Computing Platforms}
The forefront of computing platforms includes Cloud computing platforms, large multi-core and
multi-threaded processors, GPGPUs and FPGAs. These offer different degrees of general applicability,
performance and power efficiencies, and technological risk. 
Each platform exposes new
opportunities within our focal science areas, but no single platform best serves every opportunity
for accelerating scientific breakthroughs. 
Here we briefly describe the platforms and highlight
known applications. Computational technology often progresses more rapidly than science, and the
emerging computing platforms of today will form tomorrow's CI.

\subsubsection{Multi-core, Many-core and Multi-threaded processors}
We consider multi-core processors to house multiple identical processing units capable of fully
independent execution, different from some GPGPUs where cores must act in concert. 
Multi-threaded processors oversubscribe a single processing unit by appearing as many to optimize
efficient use of internal functional units. 
Since 2004, microprocessor companies have been adding
cores rather than increasing the clock speed of single cores to circumvent the ``power wall".
This
trend has permeated to general-purpose processors for laptops, servers and Cloud infrastructure, and
to embedded and accelerator processors. 
High-end commercial processors have reached ten~\cite{conf/isscc/SawantDSSRADN11} to
sixteen~\cite{Butler:2011:BAM,conf/isscc/ShinTHPPHLSJSGLS10} cores per chip for workstations and servers, and upcoming multicore accelerators
have been demonstrated with 50 or more cores~\cite{intel.mic}. 
Embedded accelerator processors that tend to
be more power-efficient and include domain-specialized instructions are up to 100 cores~\cite{Bell2008, TileGX8100}.

Multi-core processors fill an important niche in scientific processing because of their flexibility,
programmability, and ability to execute control-intensive algorithms efficiently~\cite{Kang2009,Williams2009,aparna.fmm}.
Emerging programming models are better exposing available parallelism; compilers are improving
in their ability to exploit parallelism. Domain-specific libraries and optimization strategies have
addressed applications such as protein folding~\cite{5160922} and packet analysis for network security~\cite{fusco:2010:imc}.
However, their ability to support scientific communities for grand challenges is endangered unless
software innovation finds sustainable ways to exploit the growing parallelism of multi- and manycore
processors~\cite{patternson:2010:spectrum}. As the number of cores increases, software development becomes more
challenging~\cite{conf/sc/MattsonRLBHKHVBRD10,Asanovic:2009:VPC:1562764.1562783}, especially for performance critical applications. Interconnects, topology,
and locality become more complex as the limit of cores connected via a single bus is reached.
Optimizations to limit off-chip memory bandwidth is required with multiple cores accessing remote memory. 
Reliability becomes a concern as the internal complexity and number of components in
both software and hardware increases. Debugging is also more challenging.

Multi-threaded acceleration focuses on memory bottlenecks. These accelerators tolerate latency
from accessing unpredictably scattered data by maintaining many execution threads. An execution
thread issues computational instructions once all outstanding memory references are satisfied. With
a deep enough queue of memory references and enough threads available, a processor can keep
all its execution units busy and achieve astounding efficiency. Another feature is a fine-grained
synchronization facility. 
For example, the Cray XMT~\cite{CrayInc.2007} achieves remarkable efficiency on
massive graph analysis by tolerating latencies all the way to its distributed memory units. Finegrained
locking is ``free" because it is amortized into the memory latency, and this permits automatic
optimization and software composition otherwise impossible. 
Simple software often achieves high
efficiency on multi-threaded architecture, as illustrated by the efficient analysis of twitter data
from 50,000 users for studying the spread of H1N1 influenza~\cite{Ediger2010a}. 
A more common variety of
multi-threaded acceleration is available within Intel’s Hyper-Threading (HT)~\cite{IntelCorp.2002,Riedy2011b,pasqual}. 
With
careful software design, HT has produced excellent results including the fastest, most reliable
single-node \textit{de novo} sequence assembler currently available~\cite{pasqual}.

\subsubsection{General Purpose Graphics Processing Units (GPGPUs)}
Traditionally, Graphics Processing Units (GPUs) have acted as co-processors within a computer
system providing acceleration of 3D graphics rendering. 
Recent years have brought significant
interest in exploiting the enhanced memory bandwidth and raw floating-point horsepower of GPUs
for general purpose computation; 
hence General Purpose GPUs or GPGPUs. GPGPUs' extremely
high FLOPS rates with relatively low cost and power consumption makes them attractive targets for
acceleration. 
Many scientific applications including molecular dynamics (MD) simulations~\cite{moleculardynamics} and
dense linear algebra~\cite{denselinearalgebra} successfully exploit GPGPUs. 
Several recent trends are cementing the
role of GPGPUs in future CI and improving their accessibility for scientific innovation. 
High-end GPUs are increasingly standard equipment in scientific workstations. 
Thus, many research labs
acquire GPGPUs at little or no additional cost. 
The Amazon AWS Cloud provider recently released
GPGPU VMs in their Cloud~\cite{AmazonGPU}. 
In the Top500 ranking of supercomputers, 39 systems use
GPGPUs, including three of the top five~\cite{top500:sc11}. 
All Gordon Bell Prize winners for 2009–2011 use 
GPGPUs~\cite{Hamada09,conf/sc/RahimianLVCMMSSVVZB10,Shimokawabe:2011:PPS:2063384.2063388}.

GPGPU hardware is optimized to perform massively data parallel operations on streams of data, where
a ``kernel'' operation is performed in parallel on all elements in the stream using hundreds of cores, and multiple kernels
are pipelined together.
Rebuilding software and algorithms in this form is
challenging~\cite{Yang2007:gpu}, eased by application
programming interfaces (APIs) like NVIDIA’s Compute Unified Device
Architecture (CUDA)~\cite{NVIDIA.cuda.4} and AMD’s ATI
Stream~\cite{ATI.Stream} targeted at the general purpose scientific
computing.
Algorithmic paradigm shifts are often required in existing code to
maximize the performance offered by GPGPUs.
A key strategy in improving wall clock time to scientific problem
solution recasts algorithms to reflect the underlying hardware. 
An algorithm that performs poorly on a CPU may perform orders of
magnitude better on a GPU and vice versa~\cite{Vuduc:2010kx}.

Despite being more difficult to optimize for than CPUs, GPGPUs are being
used for a variety of data-parallel computations, such as sorting, FFTs,
wavelets, and dense matrix multiply; sparse conjugate gradient and
multi-grid; bioinformatics; relational joins for databases; quantum
chemistry; and the fast multi-pole method~\cite{NVIDIA.cuda.community}.
However, productive and efficient use of these highly-capable GPGPUs
requires improved development environments that expose better
programming models, handle streaming data movement, and reduce
scheduling overheads~\cite{AugThiNamWac11CCPE}.

\subsubsection{Field Programmable Gate Arrays (FPGAs)}
Field Programmable Gate Arrays (FPGAs), invented in the mid 1980s, were predominantly used for
connecting two or more different logical subsystems. 
Recently, FPGA sizes have grown significantly
leading to heightened interest in using them as accelerators for general purpose computing. Vendors now manufacture FPGAs with multi-million, 
or even a billion, transistors. 
The large size and architectural enhancements, 
including domain-specific feature sets, provide the needed versatility
to accelerate a wide range of applications on modern FPGAs. 
Scientific applications that have leveraged accelerated kernels on FPGAs include signal
and image processing \cite{Stitt:2011:E2EFPGA}, particle physics \cite{liu:2011:hades} and network
security \cite{jiang:2011:tvlsi}.

The building blocks of reconfigurable FPGA architecture are typically an array of logic blocks,
IO pads, routing channels, DSP blocks and internal RAM. Depending on the application, these
architectural elements are programmed to perform various functions. The logic and DSP blocks can
implement arithmetic and logic operations such as add, multiply, and compare. Their results flow
through the FPGAs using configurable routing channels with point-to-point dedicated interconnects
that avoid overheads, like cache misses, present in general processors. The ability to chain thousands
of operators with data routing between them make FPGAs an excellent platform for data flow
pipelines that can accelerate scientific applications with wide parallelism. 
Some metrics of the parallelism offered include~\cite{HPCFPGA}:
\begin{itemize}
\item Internal bandwidth of Terabytes/sec (TB/s) to move results between
  application logic blocks,
\item Throughput of Tera-operations/sec (TOPS) for integer operations, and
\item Throughput of Gigaflops/sec (GFLOPS) for floating point operations.
\end{itemize}
Several FPGA-based solutions are available from
HPC system vendors \cite{FPGA-Vendors} with associated 
compiler tools, libraries and services to program these systems. While several
science communities can leverage the immense acceleration offered by FPGAs, careful
consideration has to be given to restructure their algorithms to achieve potential
gains~\cite{Storaasli2007,Fu2009,Shafer2010,Convey,Baker2007,Yi2010,Putnam2009,Ganegedara2010,Hua2008}.


\subsubsection{Cloud Computing Platforms}
The phrase ``Cloud computing" covers many levels of abstractions including virtualized dynamically
provisioned infrastructure, applications written under specific platforms and models like MapReduce,
and composable or self-contained application services. 
These infrastructure, platform, and software
as a service (IaaS, PaaS, SaaS) models are often layered. 
Cloud computing, particularly IaaS and
PaaS, is gaining support in the broader scientific community to provide on-demand and easy access
to computation and storage resources with low management overhead. 
External service providers with large data centers provision and manage elastic, 
virtual resources and lower barriers to entry.
Collocating massive datasets with large computing resources eases tackling
data-intensive grand challenge problems as demonstrated by Cloud projects
funded by DOE (Magellan~\cite{magellan}) and NSF (CluE~\cite{CluE} and
CiC~\cite{CiC}). 
Several open source Cloud fabrics~\cite{Euca, Nimbus} also exist.

Clouds require careful consideration, and our conceptual design will examine mapping applications
in biology, social science, and security to Cloud platforms. 
We will identify applications
that benefit from Clouds and also limitations in the Cloud software stack and programming models.
Clouds provide an easier platform for data collection and dissemination among large
pools of researchers and human subjects that is well suited for social sciences.
Computational biology and bioinformatics successfully leverage Cloud resources for
large scale tasks~\cite{langmead:Cloudsnp:2009, lu:azureblast:2010}.
Several efforts are on to build Science as a Service (ScaaS) for the biological
sciences community~\cite{carmen,myexperiment,Afgan2011}.

Not all science applications in our focus areas will map well to Cloud computing in their entirety.
Loosely coupled, task- and data- parallel applications are well suited for Clouds’ relatively low
network bandwidth and high latency. 
Intermittent virtual resource performance and availability
also requires application resilience. 
Some key social network analysis kernels work well in the
MapReduce model~\cite{conf/icdm/KangTF09}, but overall the model is too constrained. 
More recent generalizations map
the bulk-synchronous processing (BSP) model to Cloud resources including Google’s Pregel~\cite{conf/spaa/MalewiczABDHLC09}
for social network graph analysis, and UIUC’s Sector and Sphere~\cite{conf/spaa/MalewiczABDHLC09,oai:arXiv.org:0808.3019}, 
opening Clouds to wider use in our science domains. 
In our experience, portions of applications that extract subsets of
massive, irregularly structured datasets can take advantage of Cloud computing’s scale. 
The subset
then is passed to other computing platforms for analysis~\cite{Cloud.hybrid}. Multi-core processors and GPGPU's are
increasingly available within public Cloud
infrastructures~\cite{AmazonGPU} to support this hybrid model.
The conceptualization process, assisted by our synergistic activity~\cite{prasanna:cic:pillcrow}, will identify common
programming paradigms for utilizing Cloud resources, issues blocking wider adoption, and specific
aspects of grand challenge problems that are well suited for operating on Clouds.
