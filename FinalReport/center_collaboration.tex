\subsection{Collaboration, Education, and Outreach}
%Education, Training, Diversity, and Knowledge Transfer
%
%(3.d) Management Plan for Research, Education, Broadening Participation, and Knowledge Transfer Activities 


The fundamental requirements for sustaining a center-scale software effort are to ensure that the general community understands and uses the software developed and integrated by the center, as well as to encourage contributions to such software. Broadly we recommend three levels of collaboration and outreach activities: (a) collaborations with other major organizations that are involved in similar efforts, (b) involvement of external contributors and users through visitor programs and (c) education and training of users.

\subsubsection{Domestic and International Collaboration}
Cooperation and collaboration with other organizations involved in related and complementary efforts is valuable to the success of any center. This could be a combination of domestic collaborations with institutions funded by other agencies as well as international collaborations. International scientists might often be necessary because no single entity or country can afford a particularly expensive project or investment. Yet, S2I2 projects should not be outsourced to foreign countries. If security concerns regarding international collaborations arise, they should be handled appropriately, for example, by intellectual property or non-disclosure agreements.

\subsubsection{External Visitors}
Educating and training external visitors is a major aspect of attaining sustainability. We envision a number of different measures, ranging from personal consulting over a longer period of time (on the order of months), to medium-sized courses that take a week, to short courses and tutorials. The rationale of personal consulting is to train domain scientists on how to optimize their code for different accelerators. During their stay at the center, the guests should be expected to learn code optimization on different accelerator technologies, and subsequently share this knowledge with other members of the community, especially at their respective home institute.

The S2I2 center should aim to establish a meaningful mechanism for selecting guests to receive a substantial amount of consulting. As an example, such invitations or scholarships could be handled by a bidding mechanism or any form of peer review mechanism. It would also be beneficial to request the guests to contribute the application code developed during the visit to the center, though in some cases this might not be possible (such as for export controlled software).

Furthermore, a physical center with a core group might attract more people looking for guidance as opposed to small scattered entities. The location of this physical center might also have an impact on the number of people and length of visits the center can attract.

\subsubsection{Education and Training}
The S2I2 centers are expected to be places of expertise that can train groups of people in various parallel computing topics, with an emphasis on accelerators. While many universities educate their students in specific technologies for parallel computing, only a few of them teach a general parallel approach to algorithmics and software development; an S2I2 center would be an ideal place to take on this complementary effort. Accordingly, the center staff needs to be trained as instructors, both for large groups and for a one-to-one consulting basis. In order to attract talented researchers to the center, the teaching/consulting workload might need to be limited appropriately, such as, for example, to 50\% of their time.

Moreover, the center should educate other developers and domain scientists by defining and distributing principles and best practices for programming accelerators.
