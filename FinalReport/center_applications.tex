\subsection{Application Areas and Communities}
%(\textit{text from WSSSPE position paper})

\subsubsection{NGS-enabled bioinformatics software on emerging platforms}


\paragraph{Software infrastructure.} In addition to the need for accessible middleware and accelerated software kernels for core graph algorithms, we must also address hurdles in the full software stack: from access to specific FCPs, to accelerated graph algorithm libraries, to development platform and to distribution channels, to end user. The usability of accelerated software kernels depends on the entire software stack, and thus must be developed taking all these layers into account to achieve the portability necessary to achieve a broad user base.

For example, assembling transcriptomes directly from sequence reads without the use of a reference genome (de novo assembly) is of critical importance for any group studying gene expression in non-model organisms, and was identified as a core opportunity area during the first domain workshop. Of the popular de novo transcriptome assembly software packages, “the majority of participants agreed that Trinity \cite{trinity} was the currently the best target for development of FCP implementations, but given the rate at which assemblers come in and out of fashion, development of extensible implementations of the core graph algorithms is key.”\cite{wiki} Participants further identified efforts such as the Galaxy Project and iPlant Collaborative as potentially instrumental in connecting domain users with accelerated applications, and recommend pursuing these potential synergies. Subsequent discussions with bioinformatics tools developers, suggest supercomputing centers (eg.,Texas Advanced Computing Center or Pittsburg Supercomputing Center) or broader efforts such as XSEDE \cite{xsede} could make up the FCP layer of the software stack, and should also be pursued.

\paragraph{Supporting software sustainability.} Enabling wide-scale use of accelerated graph algorithms by NGS bioinformatics software developers, in such a way that the software and incorporation of FCP will become self-sustaining, requires developing a wide base of users. In addition to choosing the right algorithms and potential software stacks to support, our workshop participants felt that outreach and training to demonstrate the advantages of using accelerated algorithms will be key. ``Further, that the primary target for these interactions should be software developers who are tightly connected with the bio community. By creating and supporting a strong user (developer) community the software and incorporation of FCP will become self-sustaining. Suggestions as to how to maintain this developer community were to bring them together on a regular basis via, for example, yearly training workshops and hackathons.” \cite{wiki}

