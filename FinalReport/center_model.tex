\subsection{Models, Management, Organizational Structure, and Governance}
%Organizational Relationships and Reporting Structure 
%Center Management Team 
%External Advisory Committee 
Any activity undertaken by the NSF and OCI to support scientific software as a computational infrastructure will have broad importance for the science community. So it is critical that the decisions of what scientific areas, particular application codes, and computational platforms to invest in be made with vigorous, broad community input and advice from senior leadership of the community. Thus, we recommend that the S2I2 programs require that the majority of the investment (say 75\%) be prioritized by decisions made by a diverse advisory committee with a well-defined open input process from the broad scientific computing community.

However, accelerators significantly complicate the picture when considering how to select, solidify, and support scientific software. The reason for this is that today accelerators vary widely in hardware structure, software interface, and some like FPGAs present an amorphous structure to software (see earlier report section). As such, it is critical that an S2I2 center adopt a rigorous management approach to selecting which hardware accelerators will be explored and which will be supported in scientific software packages that the center supports. For example, we expect that assessments of number of platforms deployed (installed base) and future projections, expected architectural and programming model continuity of platforms, as well as support difficulty/cost would all factor into center decisions of whether or not to support an accelerator platform. In fact, it is surely the case that the supported platforms will be limited, and the specific supported platforms will change over time.

\subsubsection{Center and Management Model}
Establishment of a substantially large group at a single site for the S2I2 center structure is advised. Further, personnel from remote sites can also be included as center members, but splitting the center into many small distributed parts must be avoided to achieve the S2I2 objectives. The core group should include a full-time center director, researchers, graduate students, software developers, and supporting staff. If applicable, the director position can alternatively be held by a (preferably tenured) part-time faculty member with a full-time deputy.

The S2I2 operational model, executed by the center director, must balance research, development, and integration activities to best achieve the overall goal of software creation, maintenance, and sustainability. That is why a strong management model, which is also able to ensure the coordination of sites at different locations, is strongly preferred.

To meet the requirements of the position, an S2I2 center director must have academic credentials and have demonstrated management experience. Among his/her duties is the facilitation of coordination and interaction with federal agencies, industry, and other national labs and research centers. Also, the director is required to leverage the primary investment received from the NSF. The ability to raise funds from other sources is expected in accordance with NSF guidelines, in order to achieve sustainability beyond the NSF funding time frame.
In order to leverage a common identity both internally and externally, the S2I2 center's core group needs to be co-located at a single site with dedicated office space and other resources. Co-location with other major cyber-infrastructure entities such as a national lab is deemed unnecessary.

An S2I2 must report to an external advisory board. The board members should represent the center's target communities, including industry, SSE's and SSI's, as well as notable scientists.
The center management should identify and engage with relevant SSE’s and SSI’s. A likely incentive for an SSE/SSI to collaborate with the S2I2 is to acquire new shared funding in joint partnerships for aligned activities.

The center management is expected to focus on a small number of key projects and activities with high impact on the targeted communities. These activities not only include software development or maintenance, but also user education and training. The two latter efforts, aimed at sustainability, may include summer institutes. Detailed recommendations on education and training can be found in Section 9. Moreover, dedicated staff with expertise in software development for accelerators should provide in- depth user support for domain scientists and other users over longer periods of time, up to several months. Such support will facilitate application domain scientists’ leveraging promising new accelerator technologies, eventually autonomously. Mechanisms to ensure high scientific impact of in-depth long- term user support of specific users or projects need to be established.

\subsubsection{Governance}
The successful execution of the S2I2 and the interaction with other entities should be measured by suitable evaluation metrics. In a similar manner, the performance with respect to the identification and curation of important software artifacts should be monitored. Possible aspects to which the metrics are applied include breakthrough scientific results, publications, citations, and programmer productivity.

We recommend the NSF to encourage specific letters of collaboration from the broadest possible community of scientific software researchers, developers, and domain scientists in support of an S2I2 proposal. However, we suggest the number of letters in the supplementary documents be limited to 10.
